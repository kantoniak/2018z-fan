% Document setup
\documentclass[11pt]{article}
\usepackage[a4paper,margin=1.75cm]{geometry}

% Language and encoding
\usepackage[T1]{fontenc}
\usepackage[utf8]{inputenc}
\usepackage{polski}

% Imports
\usepackage{amsmath}
\usepackage{amssymb}
\usepackage{amsthm}
\usepackage{enumitem}
\usepackage{tikz}
\usepackage{wrapfig}
\usepackage[calc]{adjustbox}

% Styling
\usepackage{lmodern}
\setlength{\parindent}{0cm} % No paragraph indents for the first line
\usepackage{secdot} % Add dot after section number
\newcommand{\abs}[1]{\left|#1\right|} % Absolute value
\newcommand{\norm}[1]{\left\lVert#1\right\rVert} % Norm
\newcommand{\closure}[1]{\overline{#1}} % Set closure
\newcommand{\extcomplex}{\overline{\mathbb{C}}} % Extended complex plane
\newcommand{\eqtext}[1]{\overset{\text{\tiny\sffamily #1}}{=}} % Text over equality sign
\newcommand{\Forall}[1]{\mathop{\vcenter{\hbox{\LARGE$\forall$}}}\limits_{#1}} % Huge \forall for full-line rendering
\newcommand{\res}{\mathop{\text{Res}}\limits} % Residuum
\setlist[enumerate]{label=(\alph*),ref=(\alph*),leftmargin=*,topsep=0.4ex,itemsep=-0.6ex,partopsep=1ex,parsep=1ex} % List margins
\setlist[itemize]{topsep=0.4ex,itemsep=-0.6ex,partopsep=1ex,parsep=1ex} % List margins

% Theorems and definitions
\theoremstyle{plain}
\newtheorem*{theorem}{Twierdzenie}

\theoremstyle{definition}
\newtheorem*{definition}{Definicja}
\newtheorem*{corollary}{Wniosek}

\theoremstyle{remark}
\newtheorem*{remark}{Uwaga}

% Proof style
\expandafter\let\expandafter\oldproof\csname\string\proof\endcsname
\let\oldendproof\endproof
\renewenvironment{proof}[1][\proofname]{
  \oldproof[\textsc{\small #1}]
}{\oldendproof}

\renewcommand\qedsymbol{\small{$ \blacksquare $}}

% Document meta
\title{Funkcje analityczne}
\makeatletter

\begin{document}

% Title
{\huge\bfseries\@title\par}
\vspace{0.1cm}
{\Large Notatki z semestru zimowego 2018/2019\par}

% TODO: Lemat Jordana (do poprzedniej sekcji)

\section{Szeregi o wyrazach zespolonych}

\subsection{Szereg Taylora funkcji holomorficznej}

\begin{definition}
  Szereg zespolony $\sum_{n=1}^{\infty} a_{n}$ jest zbieżny do sumy $ s \in \mathbb{C} $ wtedy i tylko wtedy, gdy $ \sum_{n=1}^{k} a_{n} $ ma granicę $ s $ przy $ k\to\infty $.
\end{definition}

\begin{definition}[Jednostajna zbieżność szeregu funkcyjnego]
  Niech $ K \subset \extcomplex $ będzie dowolnym zbiorem. Niech $ f_{n} \colon K \to \mathbb{C} $, $ \norm{f}_{K} = \sup_{K}\abs{f(z)} $. Wówczas $ (f_{n}) $ jest zbieżny jednostajnie do $ f $ wtedy i tylko wtedy, gdy
  $$ \lim_{k\to\infty} \norm{f - \sum_{n=1}^{k} f_{n}}_{K} = 0. $$
\end{definition}

\begin{theorem}[Całkowanie szeregu jednostajnie zbieżnego]
  Niech $ \gamma \colon I \to \mathbb{C} $ droga kawałkami gładka, $ f_{n} \colon \gamma(I) \to \mathbb{C} $, a szereg $ \sum f_{n}(z) $ jednostajnie zbieżny na $ \gamma(I) $. Wtedy suma $ f(z) = \sum f_{n}(z) $ też jest ciągła na $ \gamma(I) $ oraz $ \int_{\gamma} f(z) dz = \sum (\int_{\gamma} f_{n}(z) dz) $.
\end{theorem}

\begin{theorem}[Kryterium Weierstrassa jednostajnej zbieżności]
  Niech $ K \subset \extcomplex $, $ f_{n} \colon K \to \mathbb{C} $. Szereg $ \sum f_{n}(z) $ jest jednostajnie zbieżny na $ K $, jeśli istnieją takie liczby $ c_{n} $, że $ \norm{f_n(z)}_K \leq c_{n} $ oraz szereg $ \sum c_{n} $ jest zbieżny.
\end{theorem}

\begin{adjustbox}{valign=C,raise=0em,minipage={1.0\linewidth}}
\setlength{\columnsep}{0.5cm}
\begin{wrapfigure}{r}{3cm}
    \vspace*{-1.5em}
    \resizebox{\linewidth}{!}{\begin{tikzpicture}
  \fill[orange!30!white] (0,0) circle (1.6);
  \draw[orange,dashed] (0,0) circle (1.6);
  \node[align=left] at (1.4,-1.4) {D};
  \fill[blue!30!white] (0,0) circle (1.2);
  \draw[blue,dashed] (0,0) circle (1.2);
  \draw[blue] (0,0) circle (0.85);
  \draw (0,0) -- (-0.7361, -0.425) node[midway, below] {\footnotesize $ r $};
  \draw (0,0) -- (1.039, -0.6) node[midway, above] {\footnotesize $ R $};
  \fill[blue] (0,0) circle (0.05) node[anchor=south, text=black] {$ a $};
\end{tikzpicture}}
\end{wrapfigure}
\strut{}
\vspace*{-1.9em}

\begin{theorem}[Rozwijanie funkcji holomorficznej w szereg Taylora]
  Niech $ f $ będzie holomorficzna w obszarze $ D $ oraz $ K(a, R) \subset D $. Wówczas istnieje szereg postaci $ \sum_{n = 0}^{\infty} c_{n}(z-a)^{n} $, zbieżny do f(z) wewnątrz koła $ K(a, R) $. Ten szereg nazywamy szeregiem Taylora.
\end{theorem}

\end{adjustbox}

\begin{proof}
  Całkujemy po koncentrycznych okręgach wewnątrz koła $ K(a, R) $.

  \begin{align*}
    f(z) &
    = \frac{1}{2 \pi i} \int_{C(a, r)} \frac{f(\zeta)}{\zeta-z} d\zeta
    = \frac{1}{2 \pi i} \int_{C(a, r)} \frac{f(\zeta)}{\zeta-a-(z-a)} d\zeta \\ &
    = \frac{1}{2 \pi i} \int_{C(a, r)} \frac{f(\zeta)}{\zeta-a} \cdot \frac{1}{1 - \frac{z-a}{\zeta-a}} d\zeta
    = \frac{1}{2 \pi i} \int_{C(a, r)} \frac{f(\zeta)}{\zeta-a} \cdot \sum_{n=0}^{\infty} \frac{(z-a)^{n}}{(\zeta-a)^{n}} d\zeta = (*)
  \end{align*}
  Korzystamy z kryterium Weierstrassa. Zauważmy, że
  $
    \abs{\frac{f(\zeta)}{\zeta-a} \cdot \frac{(z-a)^{n}}{(\zeta-a)^{n}}}
    < \frac{M}{r} \cdot \abs{\frac{z-a}{r}}^n
    < \frac{M}{r} \cdot 1
  $
  . Możemy więc całkować wyraz po wyrazie.
  \begin{align*}
    (*) &
    \eqtext{W} \sum_{n=0}^{\infty} \left( \frac{1}{2 \pi i} \int_{C(a, r)} \frac{f(\zeta)}{(\zeta-a)^{n+1}} d\zeta \right)(z-a)^{n}
    = \sum_{n = 0}^{\infty} c_{n}(z-a)^{n}
  \end{align*}
  Współczynniki $ c_{n} $ są następującej postaci:
  $$
    c_{n} = \frac{1}{2 \pi i} \int_{C(a, r)} \frac{f(\zeta)}{(\zeta-a)^{n+1}} d\zeta
  $$
\end{proof}

\begin{corollary}[Nierówność Cauchy'ego]
  Niech $ M(r) = \max_{\abs{z-a} = r < R} \abs{f(z)} $. Wówczas zachodzi nierówność
  $$ \abs{c_{n}} = \abs{ \frac{1}{2 \pi i} \int_{C(a, r)} \frac{f(\zeta)}{(\zeta-a)^{n+1}} d\zeta } \leq \frac{1}{2 \pi} \cdot \frac{M(r)}{r^{n+1}} \cdot 2 \pi r = \frac{M(r)}{r^n} $$
\end{corollary}

\begin{definition}
  Funkcja $ f $ jest całkowita wtedy i tylko wtedy gdy jest holomorficzna na $ \mathbb{C} $.
\end{definition}

\begin{theorem}[Liouville'a]
  Każda funkcja całkowita i ograniczona jest stała.
\end{theorem}

\begin{proof}
  Wnioskujemy z nierówności Cauchy'ego: wybraliśmy dowolne $ z $ i mamy $ \abs{z-a} = r > 0 $. Możemy więc wybrać taki ciąg $ z $, że $ r \to \infty $, a stąd wszystkie $ c_{n} = 0 $. 
\end{proof}

\begin{corollary}
  Każda funkcja holomorficzna na $ \extcomplex $ jest stała.
\end{corollary}

\begin{theorem}
  Niech $ c_{n} $ - dowolny ciąg liczb zespolonych. Szereg $ \sum_{n=0}^{\infty} c_{n}(z-a)^{n} $ ma promień zbieżności $ R = \frac{1}{\limsup \sqrt[n]{\abs{c_{n}}}} $, z uwzględnieniem $ 0 $ i $ \infty $.
\end{theorem}

\begin{theorem}
  Szereg potęgowy $ \sum c_{n}(z-a)^{n} $ o promieniu zbieżności $ R $ jest 
  \begin{enumerate}
    \item zbieżny dla każdej liczby $ z \in K(a, R) $; \label{prom-zbieznosci-zbieznosc}
    \item zbieżny jednostajnie na każdym zwartym podzbiorze $ K \subset K(a, R) $; \label{prom-zbieznosci-zbieznosc-jedn}
    \item rozbieżny dla każdej liczby $ z \in C \backslash \closure{K(a, R)} $. \label{prom-zbieznosci-rozbieznosc}
  \end{enumerate}
\end{theorem}

\begin{proof}
  Przyjmijmy $ 0 < R < \infty $. Wtedy $ \limsup \sqrt[n]{\abs{c_n}} = \frac{1}{R} $.

  \begin{enumerate}
    \item[\ref{prom-zbieznosci-zbieznosc}]
    Niech $ z \in K(a, R) $, wtedy $ \abs{z-a} < R $. Z def. granicy górnej 
    $ \forall \epsilon > 0 \enskip \exists n_0 \enskip \forall n \geq n_0 \enskip \sqrt[n]{\abs{c_n}} < \frac{1+\epsilon}{R} $.
    Dobierzmy taki $ \epsilon $, że $ \frac{1+\epsilon}{R} \cdot \abs{z-a} = q < 1 $.
    Wtedy $ \abs{ c_n(z-a)^n } < \frac{(1+\epsilon)^n}{R^n}(z-a)^n = q^n $ dla $ n \geq n_0 $ i szereg jest zbieżny z kryterium Weierstrassa.

    \item[\ref{prom-zbieznosci-zbieznosc-jedn}]
    Niech $ K \subset K(a, R) $ - zbiór zwarty.
    Przyjmijmy $ r = max_{z \in K}\abs{z-a} $.
    Mamy $ r < R $, bo $ K $ zwarty i ma skończone pokrycie kołami o promieniu $ \rho < R $.
    Dobieramy $ \epsilon > 0 $ tak, że $ \frac{r}{R}(1+\epsilon) = q < 1 $.
    Wtedy szereg zbieżny na K (tak jak w \ref{prom-zbieznosci-zbieznosc}), więc zbieżny jednostajnie.

    \item[\ref{prom-zbieznosci-rozbieznosc}]
    Niech $ z \in C \backslash \closure{K(a, R)} $.
    $ \forall \epsilon > 0 $ istnieje ciąg indeksów naturalnych $ (n_k) $, $ n_k \to \infty $ taki, że dla $ n \geq n_0 $ mamy $ \sqrt[n]{\abs{c_n}} > \frac{1-\epsilon}{R} $
    Dobieramy $ \epsilon $ tak/, żeby $ \frac{1-\epsilon}{R}\abs{z-a} = q > 1 $.
    Wtedy szereg jest rozbieżny, bo podciąg o indeksach $ n_k $ nie zbiega do $ 0 $.
  \end{enumerate}
\end{proof}

\begin{remark}
  Na brzegu okręgu zbieżności, $ C(a, R) $, szereg może być zbieżny lub nie w każdym punkcie pojedynczo.
\end{remark}

\subsection{Własności funkcji holomorficznych}

\begin{corollary}[Jednoznaczność rozwinięcia w szereg Taylora]
  Jeśli funkcja f jest holomorficzna w kole $ K(a, R) $ i jest zadana w tym kole zbieżnym szeregiem potęgowym $ f(z) = \sum c_n(z-a)^n $, to ten szereg jest szeregiem Taylora funkcji f.
\end{corollary}

\begin{proof}
  Rozważmy dwa szeregi o współczynnikach $ b_n $ i $ c_n $, które są rozwinięciami $ f(z) $.
  Dla każdej liczby $ k = 0, 1, 2\dots $ i każdego $ \rho \in (0, r) $ szereg $ \frac{1}{(z-a)^{k+1}} \sum_{n=0}^{\infty} c_{n}(z-a)^n = \frac{f(z)}{(z-a)^{k+1}} $ jest zbieżny jednostajnie na okręgu o promieniu $ \rho $.
  Całkujemy po tym okręgu wyraz po wyrazie i dostajemy równość $ 2 \pi i b_n = 2 \pi i c_n $, więc współczynniki muszą być równe.
\end{proof}

\begin{theorem}
  Suma szeregu potęgowego $ f(z) = \sum_{n=0}^{\infty} c_{n}(z-a)^{n} $ jest funkcją holomorficzną w swoim kole zbieżności.
  Pochodna $ f'(z) $ jest sumą szeregu otrzymanego przez różniczkowanie szeregu wyraz po wyrazie.
\end{theorem}

\begin{proof}
  Niech $ R>0 $ - promień zbieżności szeregu. Rozważmy $ g(z) = \sum_{n=1}^{\infty} nc_{n}(z-a)^{n-1} $.
  Zachodzi $ \lim_{n \to \infty} \sqrt[n]{n} = 1 $, więc $ g $ również ma promień zbieżności $ R $.

  Stąd $ g $ jest zbieżna w $ K(a, R) $ i jednostajnie zbieżna na jego podzbiorach zwartych.
  Stąd $ g $ spełnia warunki tw. o istnieniu funkcji pierwotnej:
  jest ciągła
  i całka z $ g $ po brzegu każdego trójkąta $ \subset K(a, R) $ jest zerowa (całkujemy wyraz po wyrazie, każdy z nich zerowy po brzegu z tw. Cauchy'ego).

  Funkcja pierwotna $ f_0(z) = \int_{a}^{z} g(\zeta) d\zeta $ jest holomorficzna w $ K(a, R) $ i $ f'_0 = g $.
  Całkując $ g $ wyraz po wyrazie dostajemy $ f_0 - g = c_0 $, więc f jest funkcją holomorficzną w $ K(a, R) $ i $ f' = g $.
\end{proof}

\begin{corollary}
  Niech $ f $ holomorficzna w obszarze $ D \subset \mathbb{C} $. Wówczas $ f $ ma pochodne wszystkich rzędów w $ D $.
  $ f^{(n)} $ można otrzymać przez $ n $-krotne różniczkowanie szeregu Taylora.
\end{corollary}

\begin{theorem}
  Nieh $ f(z) = \sum_{n=1}^{\infty} c_{n}(z-a)^{n} $ holomorficzna w kole $ K(a, R) $.
  Wtedy $ c_{n} = \frac{f^{(n)}(a)}{n!} $.
\end{theorem}

\begin{proof}
  Różniczkujemy szereg Taylora i podstawiamy $ z = a $.
\end{proof}

\begin{theorem}[Wzór całkowy Cauchy'ego dla pochodnych]
  Niech $ \closure{D} \subset G \subset \mathbb{C} $, gdzie $ D $, $ G $ dowolne obszary.
  Niech $ f $ holomorficzna na $ G $.
  Wtedy
  $$ \Forall{n \in \mathbb{N}_0} \Forall{a \in D} f^{(n)}(a) = \frac{n!}{2 \pi i} \int_{\partial D} \frac{f(\zeta)}{(\zeta - a)^{n+1}} d\zeta $$
\end{theorem}

\begin{proof}
  Ustalmy $ a \in D $ i niech $ \closure{K(a, r)} \subset D $, $ r < R $.
  Współczynniki $ c_n $ możemy przedstawić na dwa sposoby:
  $$
    c_n = \frac{1}{2 \pi i} \int_{C(a, r)} \frac{f(\zeta)}{(\zeta-a)^{n+1}} d\zeta
    \qquad \textnormal{oraz} \qquad
    c_n = \frac{f^{(n)}(a)}{n!}
  $$
  Po przyrównaniu stronami dostajemy
  $$
    f^{(n)} = \frac{n!}{2 \pi i} \int_{C(a, r)} \frac{f(\zeta)}{(\zeta-a)^{n+1}} d\zeta
  $$
  Na mocy tw. Cauchy'ego możemy zmienić drogę całkowania z $ C(a, r) $ na $ \partial D $.
\end{proof}

\begin{theorem}[Morery]
  Jeśli $ f $ ciągła w obszarze $ D $, oraz
  całka po brzegu dowolnego trójkąta $ \closure{\bigtriangleup} \subset D $ jest zerowa,
  to $ f $ jest holomorficzna.
\end{theorem}

\begin{proof}
  Wystarczy wykazać holomorficzność w doowlnym kole $ U \subset D $.
  Z lematu o istnieniu funkcji pierwotnej w kole wynika, że $ f $ ma funkcję pierwotną $ F $.
  Z wniosku o pochodnych szeregu Taylora $ F $ jest holomorficzna, więc $ f $ jest holomorficzna w $ U $.
\end{proof}

\begin{corollary}[Warunki równoważne holomorficzności funkcji f w punkcie $ a \in C $]
  $ $
  \begin{enumerate}
    \item $ f $ ma pochodną zespoloną w otoczeniu punktu $ a $; \label{war-holo-pochodna}
    \item $ f $ jest analityczna w $ a $, tzn. rozwija się w szereg potęgowy zbieżny w otoczeniu punktu $ a $; \label{war-holo-anal}
    \item $ f $ jest ciągła w otoczeniu $ U $ punktu $ a $ i całka z $ f $ po brzegu dowolnego trójkąta w $ U $ jest równa $ 0 $. \label{war-holo-troj}
  \end{enumerate}
\end{corollary}

\begin{proof}
  $ $
  \begin{enumerate}[leftmargin=5.1em]
    \item[\ref{war-holo-pochodna} $ \Rightarrow $ \ref{war-holo-anal}]
    Twierdzenie o rozwijaniu w szereg potęgowy;

    \item[\ref{war-holo-anal} $ \Rightarrow $ \ref{war-holo-pochodna}]
    Twierdzenie o holomorficzności sumy szeregu potęgowego;

    \item[\ref{war-holo-pochodna} $ \Rightarrow $ \ref{war-holo-troj}]
    Lemat Goursata lub twierdzenie Cauchy’ego;

    \item[\ref{war-holo-troj} $ \Rightarrow $ \ref{war-holo-pochodna}]
    Twierdzenie Morery.
  \end{enumerate}
\end{proof}

\begin{theorem}
  Niech $ f \neq const $ funkcja holomorficzna w punkcie $ a \in \mathbb{C} $.
  Niech $ f(a) = 0 $.
  Wtedy w pewnym otoczeniu $ U $ punktu $ a $ funkcję $ f $ można przedstawić jako $ f(z) = (z-a)^{n}g(z) $,
  gdzie $ g \in H(U) $, $ g(a) \neq 0 $.
\end{theorem}

\begin{proof}
  Korzystamy z postaci szeregu Taylora:
  $$ f(z) = \sum_{n=0}^{\infty} c_n(z-a)^n $$
  Mamy $ f(a) = 0 \Rightarrow c_0 = 0 $, więc z szeregu możemy wyciągnąć przynajmniej jedno $ (z-a) $.
  $$ f(z) = (z-a)^m \sum_{n=m}^{\infty} c_n(z-a)^{n-m} $$
  Dostajemy $ g(z) = \sum_{n=m}^{\infty} c_n(z-a)^{n-m} $, $ g(a) \neq 0 $.
\end{proof}

\begin{definition}
  Używaną w poprzednim twierdzeniu liczbę
  \begin{align*}
    n &
    = \min \{m \geq 1: c_m \neq 0\} \\ &
    = \min \{m \geq 1: f^{(m)}(a) \neq 0\}
  \end{align*}
  nazywamy krotnością zera $ a $ funkcji holomorficznej $ f $.
\end{definition}

\begin{corollary}
  Jeśli $ f $ jest holomorficzna w otoczeniu zawierającym $ a $, to
  albo $ f \equiv 0 $ w otoczeniu $ a $,
  albo $ f(z) \neq 0 $ w pewnym otoczeniu $ a $.
\end{corollary}

% TODO: Dowód

\begin{theorem}[O jednoznaczności]
  Niech $f$, $g$ holomorficzne w obszarze $D$.
  Niech $(z_n)$ pewien ciąg o wyrazach w $D$, mający punkt skupienia $a$ w $D$.
  Niech $ f(z_n) = g(z_n) $.
  Wówczas $f$ i $g$ są identyczne na $D$.
\end{theorem}

\begin{proof}
  Niech $ U \subset D $ pewne otoczenie $ a $.
  Niech $ A = \{ z \in U: f(z) = g(z) \} $.
  $ A $ jest otwarty (z definicji). % TODO: Dlaczego dokładnie?
  $ A $ jest domknięty (z poprzedniego wniosku jeśli $ a_0 \in D $ punkt skupienia $ A $, to $ f(a_0) = g(a_0) \Rightarrow a_0 \in A $).
  Punkt skupienia $ a \in A $, więc $ A $ niepusty.
  $ D $ jest spójny, więc $ A = D $.
\end{proof}

\begin{theorem}[Zasada maksimum]
  Niech $ f $ funkcja holomorficzna w obszarze $ D $.
  Jeśli $ \max_D \abs{f(z)} $ jest osiągane wewnątrz $ D $, to $ f $ jest stała.
\end{theorem}

\begin{proof}
  Niech $ \abs{f(a)} = g = \sup_{z \in D} \abs{f(z)} $. Całkujemy po okręgach w otoczeniu $ a $.
  \begin{align*}
    0 &
    = \abs{f(a)} - g
    = \abs{ \frac{1}{2 \pi} \int_{0}^{2 \pi} f(a + re^{it}) dt } - g
    \leq \frac{1}{2 \pi} \int_{0}^{2 \pi} \abs{f(a + re^{it})} dt - g
    \leq \frac{1}{2 \pi} \int_{0}^{2 \pi} g dt - g = 0
  \end{align*}
  Stąd mamy
  \begin{align*}
    0 &
    = g - \frac{1}{2 \pi} \int_{0}^{2 \pi} \abs{f(a + re^{it})} dt
    = \frac{1}{2 \pi} \int_{0}^{2 \pi} (g - \abs{f(a + re^{it})}) dt
  \end{align*}
  Funkcja podcałkowa jest ciągła i nieujemna, więc żeby zachować równość musi być $ g = \abs{f(a +re^{it})} $ na całym okręgu.

  Z dowolności $ r $ na całym kole jest $ g = \abs{f(z)} $.
  Żeby zachodziły warunki Cauchy’ego-Riemanna, to $ f $ musi być stała na $ D $.
  Z tw. o jednoznaczności $ f(z) = g $ na całym $ D $.
\end{proof}

\begin{corollary}
  Funkcja $f$ holomorficzna w obszarze ograniczonym $D$ i ciągła w $D$ osiąga maksimum modułu na brzegu $\partial D$ obszaru $D$.
\end{corollary}

\begin{proof}
  Dla funkcji stałej teza zachodzi.
  Niech $ f \neq const $.
  Wtedy $ \max\abs{f(z)} $ nie należy do wnętrza $D$, więc leży na brzegu.
\end{proof}

\begin{theorem}[Lemat Schwarza]
  Niech $ f $ funkcja holomorficzna w kole $ K(0, R) $.
  Jeśli $ f(0) = 0 $ oraz $ \forall z \in D\enskip\abs{f(z)} \leq M $, to
  $$
    \abs{f'(0)} \leq \frac{M}{R}
    \qquad \textnormal{oraz} \qquad
    \abs{f(z)} \leq \frac{M}{R}\abs{z}
  $$
  Ponadto, równość zachodzi wtedy i tylko wtedy, gdy $ f $ jest postaci
  $$
    f(z) = \frac{M}{R}e^{it}z
  $$
\end{theorem}

\begin{proof}
  Bez utraty ogólności możemy dowodzić dla $ M = 1 $, $ R = 1 $.

  Z $ f(0) = 0 \Rightarrow f(z) = z \cdot g(z) $, gdzie $g$ jest holomorficzna w kole $ K(0,1) $.
  Stosujemy zasadę maksimum dla $ g(z) $, $ r < R = 1 $.
  $$
    \max_{z \in \closure{K(0, r)}} \abs{g(z)} =
    \max_{z \in C(0, r)} \abs{g(z)}
    \leq \max_{z \in C(0, r)} \frac{\abs{f(z)}}{\abs{z}}
    \leq \frac{1}{r}
  $$
  Przy $ r \to 1 $ dostajemy $ max_{z \in \closure{K(0, r)}} \abs{g(z)} \leq 1 $.
  Stąd
  $$
    \Forall{z \in K(0, r)} 1 \geq \abs{g(z)}
    = \frac{\abs{f(z)}}{\abs{z}}
    \qquad \implies \qquad
    \abs{f(z)} \leq \abs{z}
  $$
  co dowodzi jednej nierówności z twierdzenia.
  Dla pochodnej mamy
  $$
    f(z) = zg(z)
    \quad\Rightarrow\quad
    f'(z) = g(z) + zg'(z)
    \quad\Rightarrow\quad
    f'(0) = g(0)
    \quad\Rightarrow\quad
    \abs{f'(0)} = \abs{g(0)}
    \leq 1
  $$

  Jeśli $ \abs{f(z_0)} = \abs{z_0} $ dla pewnego $ z_0 \neq 0 $, to $ \abs{g(z_0)} = 1 $.
  Czyli $ g $ osiąga maksimum wewnątrz $ K(0, 1) $, więc z zasady maksimum jest stała.
  Skoro $ \abs{g(z)} = 1 $, to $ g(z) = e^{it} $.
\end{proof}

\begin{definition}[Niemal jednostajna zbieżność szeregu funkcyjnego]
  Szereg $ \sum f_{n} $ jest zbieżny niemal jednostajnie na obszarze $D$ wtedy i tylko wtedy gdy jest zbieżny jednostajnie na każdym zwartym podzbiorze $D$.
\end{definition}

\begin{theorem}[Weierstrassa o szeregach funkcji holomorficznych]
  Niech szereg funkcji holomorficznych (na $D$) $ \sum_{n=1}^{\infty} f_n(z) $ będzie niemal jednostajnie zbieżny na obszarze $D$. Wtedy

  \begin{enumerate}
    \item $ f(z) = \sum_{n=1}^{\infty} f_n(z) $ jest holomorficzna na $ D $; \label{weierstrass-holo-f}
    \item $ \forall k \in \mathbb{N}\enskip f^{(k)}(z) = \sum_{n=1}^{\infty} f_n^{(k)}(z) $; \label{weierstrass-holo-pochodne}
    \item $ \forall k \in \mathbb{N}\enskip \sum_{n=1}^{\infty} f_n^{(k)}(z) $ zbieżny niemal jednostajnie na $D$; \label{weierstrass-holo-pochodne-zb}
  \end{enumerate}
\end{theorem}

\begin{proof}
  Rozważmy dwa koła wokół dowolnego punktu $a$.
  Koła są współśrodkowe i zawierają się w $D$.
  $ U = K(a, r) $, $ V = K(a, R) $, $ 0 < r < R $, $ \closure{U} \subset V $, $ \closure{V} \subset D $.

  \begin{enumerate}
    % TODO: Obrazek

    \item[\ref{weierstrass-holo-f}]
    Rozważamy tylko pojedyncze koło $U$.
    Z niejednostajnej zbieżności f jest ciągła i zbieżna jednostajnie na zwartym $ \closure{U} $.
    Stąd dla każdego trójkąta $ \closure{\bigtriangleup} \subset U $ mamy
    $$
      \int_{\partial \bigtriangleup} f(z)dz =
      \sum_{n=1}^{\infty}\int_{\partial \bigtriangleup} f_n(z)dz =
      \sum_{n=1}^{\infty} 0 =
      0
    $$
    Stąd z tw. Morery $ f \in H(U) $, a z dowolności $ U $ mamy $ f \in H(D) $.

    \item[\ref{weierstrass-holo-pochodne}]
    Korzystamy dwukrotnie ze wzoru całkowego Cauchy'ego dla pochodnych:
    \begin{align*}
      \Forall{k \in \mathbb{N}} f^{(k)}(z) &
      \eqtext{C} \frac{k!}{2 \pi i} \int_{\partial V} \frac{f(\zeta)}{(\zeta - z)^{k+1}} d\zeta \\ &
      \eqtext{Z} \sum_{n=1}^{\infty} \frac{k!}{2 \pi i} \int_{\partial V} \frac{f_n(\zeta)}{(\zeta - z)^{k+1}} d\zeta \\ &
      \eqtext{C} \sum_{n=1}^{\infty} f_n^{(k)}(z)
    \end{align*}
    Przejście Z możemy zastosować bo mianownik w funkcji podcałkowej jest ogarniczony z dołu, więc cały ułamek zbieżny jednostajnie.

    \item[\ref{weierstrass-holo-pochodne-zb}]
    Pokażemy, że $ \sum_{n=1}^{\infty} f_n^{(k)}(z) $ jest zbieżny jednostajnie na kole $U$.
    Szacujemy resztę $ r_n(\zeta) = f(\zeta) - \sum_{j=1}^{n} f_j(\zeta) $
    Korzystamy ze wzoru całkowego dla pochodnych:
    $$
      \abs{ f^{(k)}(z) - \sum_{j=1}^{n} f_j(z) }
      = \frac{k!}{2 \pi} \abs{ \int_{\partial V} \frac{r_n(\zeta)}{(\zeta - z)^{(k+1)}} d\zeta }
      \leq \frac{k!}{2 \pi} \cdot \frac{ \max_{\zeta \in \partial V} \abs{r_n(\zeta)} }{ (R - r)^{(k+1)} } \cdot 2 \pi R
    $$
    Końcowe wyrażenie nie zależy od $z$ i dąży do $0$ w nieskończoności.

    $U$ jest zwarty, więc ma pokrycie skończone kołami $ U_1 $, $ \dots $, $ U_s $.
    Szereg jest zbieżny na każdym z nich, więc na ich sumie też, więc również na $U$.
  \end{enumerate}  
\end{proof}

\begin{theorem}[O ciągach funkcji holomorficznych]
  Niech $ (f_n(z)) $ ciąg funkcji holomorficznych na $D$, niemal jednostajnie zbieżny na $D$. Wtedy

  \begin{enumerate}
    \item Granica $ f(z) = \lim_{n \to \infty} f_n(z) $ jest holomorficzna na $ D $; \label{ciagi-holo-f}
    \item $ \forall k \in \mathbb{N}\enskip f^{(k)}(z) = \lim_{n \to \infty} f_n^{(k)}(z) $; \label{ciagi-holo-pochodne}
    \item $ \forall k \in \mathbb{N}\enskip (f_n^{(k)}(z)) $ zbieżny niemal jednostajnie na $D$; \label{ciagi-holo-pochodne-zb}
  \end{enumerate}
\end{theorem}

\subsection{Szereg Laurenta i otoczenia pierścieniowe}

\pagebreak

\begin{theorem}[O rozwijaniu funkcji holomorficznej w szereg Laurenta]
  Rozważmy pierścień $ V = \{z \in \mathbb{C}\colon r < \abs{z-a} < R\} $, $ 0 \leq r < R \leq +\infty $.
  Niech $ f $ funkcja holomorficzna na $ V $.
  Oznaczmy
  $$
    c_n = \frac{1}{2 \pi i} \int_{C(a, \rho)} \frac{f(\zeta)}{(\zeta - a)^{n+1}} d\zeta,
    \quad
    n \in \mathbb{Z},
    \quad
    r < \rho < R
  $$
  Wtedy

  \begin{itemize}
    \item $ c_n $ nie zależy od $ \rho $. Liczby $ c_n $ nazywamy współczynnikami szeregu Laurenta;
    \item Szereg Laurenta, $ \sum_{n = -\infty}^{\infty} c_n(z-a)^n $, jest zbieżny do $ f(z) \enskip \forall z \in V $.
  \end{itemize}

  Szereg Laurenta dzielimy na część główną i regularną:
  $$
    f(z) =
    \sum_{n = -\infty}^{\infty} c_n(z-a)^n =
    \underbrace{\sum_{n = -\infty}^{-1} c_n(z-a)^n}_\text{\normalfont\tiny\sffamily część główna} +
    \underbrace{\sum_{n = 0}^{\infty} c_n(z-a)^n}_\text{\normalfont\tiny\sffamily część regularna}
  $$
\end{theorem}

\begin{proof}
  Korzystamy z tw. Cauchy'ego.
  Niech $ z \in V $ dowolny punkt.
  Rozważamy pierścień $ z \in U \subset V $ o promieniach $s$, $t$.
  Korzystamy ze wzoru całkowego na $ U $: 

  \begin{adjustbox}{valign=C,raise=0em,minipage={1.0\linewidth}}
  \setlength{\columnsep}{0.5cm}
  \begin{wrapfigure}{r}{3cm}
      \vspace*{-1.5em}
      \resizebox{\linewidth}{!}{\begin{tikzpicture}
  \fill[orange!30!white] (0,0) circle (1.6);
  \draw[orange,dashed] (0,0) circle (1.6);

  \fill[blue!30!white] (0,0) circle (1.4);
  \draw[blue,dashed] (0,0) circle (1.4);

  \fill[orange!30!white] (0,0) circle (0.6);
  \draw[blue,dashed] (0,0) circle (0.6);

  \fill[white] (0,0) circle (0.4);
  \draw[orange,dashed] (0,0) circle (0.4);

  \draw (0,0) -- (-0.2828, 0.2828) node[midway, above, xshift=2, yshift=-2.2] {\footnotesize $ r $};
  \draw (0,0) -- (-1.4782, 0.6123) node[midway, above] {\footnotesize $ R $};
  \draw (0,0) -- (0.4243, 0.4243) node[midway, above, xshift=-2, yshift=-1.8] {\footnotesize $ s $};
  \draw (0,0) -- (1.2934, 0.5356) node[midway, above] {\footnotesize $ t $};

  \fill[darkgray] (0,0) circle (0.05) node[anchor=north, text=black] {$ a $};
  \fill[darkgray] (-0.9,-0.6) circle (0.05) node[anchor=west, text=black] {$ z $};

  \node[align=left] at (0.75,-0.75) {U};
  \node[align=left] at (1.4,-1.4) {V};
\end{tikzpicture}}
  \end{wrapfigure}
  \strut{}
    \vspace*{-1em}
  
    $$
      f(z)
      = \frac{1}{2 \pi i} \int_{C(a, t)} \frac{f(\zeta)}{\zeta - z} d\zeta
      - \frac{1}{2 \pi i} \int_{C(a, s)} \frac{f(\zeta)}{\zeta - z} d\zeta
      = I_1 - I_2
    $$

    \vspace*{0.5em}

    Obliczamy całkę $ I_1 $.
    Mamy $ \abs{z-a} < \abs{\zeta - a} $ dla $ \zeta \in C(a, t) $, więc możemy rozwijać w szereg potęgowy:
  
  \end{adjustbox}

  \begin{align*}
    \frac{f(\zeta)}{\zeta - z}
    = \frac{f(\zeta)}{(\zeta - a) - (z - a)}
    = \frac{f(\zeta)}{\zeta - a} \cdot \frac{1}{1 - \frac{z - a}{\zeta - a}} 
    = \frac{f(\zeta)}{\zeta - a} \cdot \sum_{n=0}^{\infty} \frac{(z - a)^n}{(\zeta - a)^n}
    = \sum_{n=0}^{\infty} \frac{f(\zeta)(z - a)^n}{(\zeta - a)^{n+1}}
    \label{eq:laurent-int-expansion} \tag{\small $\spadesuit$}
  \end{align*}

  Skadniki sumy możemy oszacować z góry jako
  $$
    \abs{ \frac{f(\zeta)(z - a)^n}{(\zeta - a)^{n+1}} }
    \leq \frac{M(t)}{t}\left(\frac{\abs{z-a}}{t}\right)^n,
    \qquad
    M(t) = \max_{\zeta \in C(a, t)} \abs{f(\zeta)}
  $$
  więc szereg \eqref{eq:laurent-int-expansion} jest zbieżny jednostajnie ze względu na $ \zeta $.
  Stąd można całkować \eqref{eq:laurent-int-expansion} wyraz po wyrazie po $ C(a, t) $.

  Obliczamy całkę $ I_2 $.
  Teraz $ \abs{z-a} > \abs{\zeta - a} $ dla $ \zeta \in C(a, s) $ i rozwijamy odwrotnie:
  $$
    \frac{f(\zeta)}{\zeta - z}
    = \frac{f(\zeta)}{(\zeta - a) - (z - a)}
    = \frac{f(\zeta)}{z - a} \cdot \frac{1}{\frac{\zeta - a}{z - a} - 1}
    = -\frac{f(\zeta)}{z - a} \cdot \sum_{m=0}^{\infty} \frac{(\zeta - a)^m}{(z - a)^m}
    = -\sum_{m=0}^{\infty} \frac{f(\zeta)(\zeta - a)^m}{(z - a)^{m+1}}
  $$
  Całkując wyraz po wyrazie po $ C(a, s) $ dostajemy:
  $$
    I_2 = 
    \sum_{m=0}^{\infty} b_m(z-a)^{-(m+1)},
    \qquad
    b_m =
    -\frac{1}{2 \pi i} \int_{C(a,s)} (\zeta - a)^{m} f(\zeta) d\zeta
    = -c_{-(m+1)}
  $$
  Więc całka $ I_2 $ jest równa części głównej szeregu Laurenta.
\end{proof}

\begin{theorem}
  Niech $f$, $g$ fukcje holomorficzne w pierścieniu $V$.
  Niech $ f(z) = \sum_{n=-\infty}^{\infty} b_n(z-a)^n $, $ g(z) = \sum_{n=-\infty}^{\infty} c_n(z-a)^n $.
  Wtedy szeregi $ d_n = \sum_{m=-\infty}^{\infty} b_{m}c_{n-m} = \sum_{m=-\infty}^{\infty} c_{m}b_{n-m} $ są zbieżne $ \forall n \in \mathbb{Z} $
  oraz $ f(z) \cdot g(z) = \sum_{n=-\infty}^{\infty} d_n(z-a)^n $.
\end{theorem}

\begin{proof}
  Funkcja $ fg $ jest holomorficzna w $V$, więc jest sumą pewnego szeregu Laurenta w $V$.
  Współczynniki są równe
  $$
    d_n =
    \frac{1}{2 \pi i} \int_{C(a, \rho)} \frac{f(\zeta)g(\zeta)}{(\zeta - a)^{n+1}} d\zeta,
    \qquad
    r < \rho < R
  $$

  Na $ C(a, \rho) $ szereg Laurenta $ f $ jest do niej jednostajnie zbieżny, więc możemy podstawić szereg pod $f$:
  \begin{align*}
    d_n &
    = \frac{1}{2 \pi i} \int_{C(a, \rho)} \left[ \sum_{m=-\infty}^{\infty} b_m(\zeta - a)^m \right] g(\zeta) (\zeta - a)^{-n-1} d\zeta \\ &
    = \sum_{m=-\infty}^{\infty} b_m \left[ \frac{1}{2 \pi i} \int_{C(a, \rho)} g(\zeta) (\zeta - a)^{m-n-1} d\zeta \right] \\ &
    = \sum_{m=-\infty}^{\infty} b_{m}c_{n-m}
  \end{align*}

  Dowód dla drugiej sumy analogiczny, podstawiamy pod $g$.
\end{proof}

\begin{theorem}
  Niech $ n \in \mathbb{Z} $ i $ c_n $ dowolne liczby zespolone.
  Weźmy $ r = \limsup \sqrt[n]{c_{-n}} $, $ R = \frac{1}{ \limsup \sqrt[n]{c_{-n}} } $.
  Wówczas szereg Laurenta $ f(z) = \sum_{n=-\infty}^{\infty} c_n(z-a)^{n} $ jest zbieżny bezwzględnie i niemal jednostajnie w pierścieniu $ \{ z \in \mathbb{C}: r < \abs{z-a} < R \} $,
  $ f(z) $ jest holomorficzna w tym pierścieniu oraz
  $$
    c_n = \frac{1}{2 \pi i} \int_{C(a, \rho)} \frac{f(\zeta)}{(\zeta - a)^{n+1}} d\zeta,
    \quad
    n \in \mathbb{Z},
    \quad
    r < \rho < R
  $$

  Jeśli $ \abs{z-a} < r $, to część główna jest rozbieżna.
  Jeśli $ \abs{z-a} > R $, to część regularna jest rozbieżna. 
\end{theorem}

\begin{proof}
  Zbieżność dowodzimy z tw. Cauchy'ego-Hadamarda. Rozważamy dwa szeregi i patrzymy na ich promienie zbieżności:
  $$
    f_1(z) = \sum_{n=0}^{\infty} c_n(z-a)^{n},
    \quad
    f_2(z) = \sum_{n=-\infty}^{-1} c_n(z-a)^{n} = \sum_{m=1}^{\infty} c_n \left(\frac{1}{z-a}\right)^{m} = \sum_{m=1}^{\infty} c_n Z^m
  $$
  Funkcja $ f_1 $ holomorficzna w kole $ \abs{z-a} < R $, $ f_2 $ w $ \abs{Z} < r^{-1} $, tzn. $ \abs{z-a} > r $.

  % TODO: Dowód na postać c_n
\end{proof}

\begin{remark}
  Dla funkcji holomorficznej w pierścieniu zachodzą nierówności Cauchy'ego, tak jak na kole.
\end{remark}

% TODO: Uwaga 6 z wykładu 8 w notatkach (o szeregu Fouriera)

\subsection{Punkty osobliwe}

\begin{definition}
  Punkt $ a \in \extcomplex $ nazywa się punktem osobliwym funkcji $f$ wtedy i tylko wtedy,
  gdy funkcja $f$ nie jest holomorficzna w $a$,
  a w każdym otoczeniu punktu $a$ istnieje punkt, w którym funkcja $f$ jest holomorficzna.
\end{definition}

\begin{remark}
  Mowa jest tu o funkcjach przyjmujących wartości w $ \mathbb{C} $, a nie o przekształceniach z wartościami w $ \extcomplex $.
  Jeśli nawet funkcja $ f: U \to \mathbb{C} $ rozszerza się do przekształcenia $ F: U \to \extcomplex $ holomorficznego w $a$ i takiego,
  że $ F(a) = \infty $, to tym niemniej punkt $a$ jest punktem osobliwym funkcji $f$.
\end{remark}

\begin{definition}[Izolowane punkty osobliwe]
  Punkt $ a \in \mathbb{C} $ nazywa się izolowanym punktem osobliwym funkcji $f$,
  jeśli funkcja $f$ jest holomorficzna w pewnym otoczeniu pierścieniowym $ V = \{ z \in \mathbb{C}: 0 < \abs{z-a} < \epsilon \} $, gdzie $ \epsilon > 0 $.
  Izolowany punkt osobliwy $a$ funkcji $f$ nazywa się:

  \begin{enumerate}
    \item punktem pozornie osobliwym, jeśli istnieje skończona granica $ \lim_{z \to a} f(z) \in \mathbb{C} $;
    \item biegunem, jeśli istnieje granica $ \lim_{z \to a} f(z) = \infty $;
    \item punktem istotnie osobliwym, jeśli nie istnieje granica $ \lim_{z \to a} f(z) $.
  \end{enumerate}
\end{definition}


\begin{theorem}[Riemanna]
  Niech $f$ funkcja holomorficzna w pewnym otoczeniu pierścieniowym $ V = \{ z \in \mathbb{C}: 0 < \abs{z-a} < \epsilon \} $.
  Następujące warunki są równoważne:

  \begin{enumerate}
    \item $ a $ jest punktem pozornie osobliwym funkcji $ f $; \label{tw-riemanna-poz}
    \item $ f $ jest ograniczona w pewnym otoczeniu pierścieniowym  $ V' = \{ z \in \mathbb{C}: 0 < \abs{z-a} < \epsilon' \} $; \label{tw-riemanna-ogr}
    \item część główna szeregu Laurenta jest zerowa; \label{tw-riemanna-gl-zerowa}
    \item można określić wartość $ f(a) $ w taki sposób, aby otrzymać funkcję holomorficzną w całym kole $ K(0, \epsilon) $. \label{tw-riemanna-dookreslenie}
  \end{enumerate}
\end{theorem}

\begin{proof}
  $ $
  \begin{enumerate}[leftmargin=5.1em]
    \item[\ref{tw-riemanna-poz} $ \Rightarrow $ \ref{tw-riemanna-ogr}]
    Z definicji

    \item[\ref{tw-riemanna-ogr} $ \Rightarrow $ \ref{tw-riemanna-gl-zerowa}]
    Korzystamy z nierówności Cauchy’ego:
    $$
      c_{n} =  \frac{1}{2 \pi i} \int_{C(a, \rho)} \frac{f(\zeta)}{(\zeta-a)^{-n+1}} d\zeta
      \qquad\implies\qquad
      \abs{ c_n } \leq \frac{M}{\rho^{-n}} = M\rho^n \to 0
    $$
    Stąd $ \abs{c_n} = 0 $.

    \item[\ref{tw-riemanna-gl-zerowa} $ \Rightarrow $ \ref{tw-riemanna-dookreslenie}]
    $ f $ jest opisana szeregiem Taylora $ f(z) = \sum_{n=0}^{\infty} c_n(z-a)^n $.
    Możemy dookreślić $ f(a) = c_0 $ i wtedy szereg działa również w punkcie $ z = a $.

    \item[\ref{tw-riemanna-dookreslenie} $ \Rightarrow $ \ref{tw-riemanna-poz}]
    Oczywistość
  \end{enumerate}
\end{proof}

\begin{theorem}
  Niech $f$ funkcja holomorficzna w pewnym otoczeniu pierścieniowym $ V = \{ z \in \mathbb{C}: 0 < \abs{z-a} < \epsilon \} $.
  Punkt $a$ jest biegunem funkcji $f$ wtedy i tylko wtedy,
  gdy część główna szeregu Laurenta $ f $ jest skończona, ale niezerowa.
\end{theorem}

\begin{proof}
  ($\Leftarrow$) Oczywistość

  ($\Rightarrow$) Z definicji bieguna $ \lim_{z \to a} f(z) = \infty $, stąd $ f(z) \neq 0 $ w pierścieniu $ V' = \{ z \in \mathbb{C}: 0 < \abs{z-a} < \epsilon' \} $.
  Stąd $ g(z) = \frac{1}{f(z)} $ holomorficzna w $ V' $ oraz $ \lim_{z \to a} g(z) = 0 $.

  Z twierdzenia Riemanna możemy uzupełnić $ g(a) = 0 $ i wtedy $ g $ holomorficzna w kole $ U' = K(a, \epsilon') $.
  Niech $ N $ krotność zera $g$ w $a$.
  Wtedy $  g(z) = (z-a)^{N}h(z) $, gdzie $h$ holomorficzna na $ U' $ i $ h(z) \neq 0 $ na $ U'' = K(a, \epsilon'') $.
  Stąd $ \frac{1}{h(z)} $ rozwija się w pewien szereg $ \frac{1}{h(z)} = b_0 + b_1(z-a) + \dots $ na $ U'' $.
  Stąd $ f(z) = \frac{b_0 + b_1(z-a) + \dots}{(z-a)^{N}} $ na pierścieniu $ V'' $.
\end{proof}

\begin{remark}
  Liczba $ N $ j.w. nazywa się krotnością bieguna funkcji $f$ w $a$ i jest równa krotności zera $a$ funkcji $ \frac{1}{f(z)} $ w punkcie $ a $.
  Wtedy $ c_{-N} \neq 0 $ oraz $c_k = 0$ dla $ k < -N $.
\end{remark}

\begin{corollary}
  Funkcja $f$ holomorficzna w otoczeniu pierścieniowym $ V = \{ z \in \mathbb{C}: 0 < \abs{z-a} < \epsilon \} $ ma w tym punkcie osobliwość istotną wtedy i tylko wtedy,
  gdy istnieje nieskończenie wiele liczb całkowitych $ n \geq 1 $ takich, że $ c_{-n} \neq 0 $.
\end{corollary}

\begin{theorem}[Casoratiego-Weierstrassa]
  Niech $ a\in \mathbb{C} $ punkt istotnie osobliwy funkcji $ f $.
  Niech $ A \in \extcomplex $.
  Wówczas istnieje ciąg $ z_n \to a $ taki, że $ \lim_{n \to \infty} f(z_n) = A $.
\end{theorem}

\begin{proof}
  ($ A = \infty $) $f$ nie jest ograniczona w żadnym otoczeniu punktu $a$, więc wybieramy punkty o coraz większych wartościach.

  ($ A \in \mathbb{C} $) Jeśli w każdym otoczeniu $a$ istnieje punkt $ z $ taki, że $ f(z) = A $, to wybraliśmy ciąg.
  Jeśli nie, to funkcja $ g(z) = \frac{1}{f(a) - A} $ ma w $ a $ izolowany punkt osobliwy.
  Nie jest on ani pozorny, ani biegunem, bo wtedy $f$ miałaby granicę w $a$.
  Stąd $a$ jest punktem istotnie osobliwym funkcji $g$ i stosujemy pierwszą część dowodu do $g$.
  A gdy $ g(z_n) \to \infty $, to $ f(z_n) \to A $.
\end{proof}

\begin{remark}
  Z definicji punktów osobliwych wynika, że $ \infty $ jest punktem pozornie osobliwym, biegunem, punktem istotnie osobliwym funkcji $f$ wtedy i tylko wtedy,
  gdy punkt $0$ jest takim punktem dla funkcji $ f(\frac{1}{z} $. W terminach szeregu Laurenta wygląda to tak.
\end{remark}

\begin{theorem}
  Niech $ f(z) = \sum_{n=-\infty}^{\infty} c_n z^n $, $ \abs{z} > R $.
  Wtedy punkt $ a = \infty $ jest:
  \begin{enumerate}
    \item punktem pozornie osobliwym $ f \Leftrightarrow c_n = 0 \enskip\forall n \geq 1 $; \label{osobliwy-infty-poz}
    \item biegunem $ f \Leftrightarrow \exists N \geq 1 $ taka, że $ c_N \neq 0 $ i $ c_n = 0 \enskip\forall n \geq N + 1 $; \label{osobliwy-infty-bieg}
    \item  punktem istotnie osobliwym funkcji $ f \Leftrightarrow c_n \neq 0 $ dla nieskończenie wielu $ n \geq 1 $. \label{osobliwy-infty-ist}
  \end{enumerate}
  W związku z tym częścią główną szeregu Laurenta funkcji $ f $ w otoczeniu pierścieniowym nieskończoności nazywa się szereg $ \sum_{n=1}^{\infty} c_n z^n $, a częścią regularną szereg $ \sum_{n=-\infty}^{0} c_n z^n $
\end{theorem}

\begin{corollary}
  Jeśli funkcja całkowita $f$ ma w $ \infty $ punkt pozornie osobliwy lub biegun, to jest wielomianem.
\end{corollary}

\begin{proof}
  Niech $ P(z) = \sum_{n=1}^{\infty} c_n z^n $ część główna szeregu Laurenta w otoczeniu pierścieniowym punktu $ \infty $.
  Wtedy $ P(z) $ jest wielomianem.
  Wówczas $ g(z) = f(z) - P(z) $ jest całkowita i ma osobliwość pozorną w $ \infty $,
  a stąd jest stała z tw. Liouville'a.
\end{proof}

\begin{definition}
  Funkcja $f$ nazywa się meromorficzna w obszarze $ D \subset \extcomplex $ jeśli jedynymi jej punktami osobliwymi są (punkty pozornie osobliwe i) bieguny.
  Z definicji bieguna wynika, że jest on izolowany, więc jest ich co najwyżej przeliczalnie wiele (bo skończenie wiele w każdym zbiorze zwartym).
\end{definition}

\begin{theorem}
  Jeśli funkcja $f$ jest meromorficzna w $ \extcomplex $, to jest funkcją wymierną.
\end{theorem}

\begin{proof}
  Ponieważ bieguny w $ \extcomplex $ są izolowane, to jest ich skończenie wiele.
  Oznaczmy je $ a_1, \dots, a_n $.
  Niech $ R_j(z) = \sum_{k=1}{n_j} c_{k} (z-a)^{-k} $, $ j = 1, \dots, n $ oznacza część główną szeregu Laurenta $ f $ w otoczeniu pierścieniowym punktu $ a_j $,
  a $ P(z) = \sum_{k=1}^{m} c_k z^k $ część główną w otoczeniu pierścieniowym $ \infty $.
  Wtedy funkcja $ g(z) = f(z) - (P(z) + R_1(z) + \dots + R_n(z)) $ jest holomorficzna w $ \mathbb{C} $ i ma osobliwość pozorną w $ \infty $,
  więc jest stała na mocy tw. Liouville'a.
\end{proof}

\begin{definition}
  Niech $ f $ będzie funkcją holomorficzną w pierścieniowym otoczeniu $ V = \{z \in \mathbb{C}\colon 0 < \abs{z-a} < \epsilon \} $ punktu $ a \in \mathbb{C} $.
  Residuum funkcji $ f $ w punkcie $ a $ nazywamy liczbę
  $$
    \res(f, a) = \frac{1}{2 \pi i} \int_{C(a, r)} f(\zeta) d\zeta,
    \qquad
    0 < r < \epsilon
  $$
\end{definition}

\begin{theorem}[Cauchy’ego o residuach]
  Niech $ D \subset \extcomplex $ obszar ograniczony skończoną liczbą konturów zawartych z brzegiem w obszarze $ G $.
  Niech $ f $ funkcja holomorficzna w $ G $ poza skończoną liczbą punktów osobliwych $ a_1, \dots, a_n \in D $.
  Wtedy
  $$
    \int_{\partial D} f(\zeta) d\zeta = 2 \pi i \sum_{j=1}^n \res(f, a)
  $$ 
\end{theorem}

\begin{proof}
  Dobieramy $ \epsilon > 0 $ tak, żeby koła $ B_j = \{ z \in \mathbb{C}: \abs{z - a_j} < \epsilon \} $ były parami rozłączne.
  Wtedy $ D_{\epsilon} = D\backslash\bigcup\limits_{j=1}^{n} B_j $ jest obszarem wielospójnym i możemy zastosować tw. Cauchy'ego:
  $$
    0
    = \int_{\partial D_{\epsilon}} f(\zeta) d\zeta
    = \int_{\partial D} f(\zeta) d\zeta - \sum_{j=1}^{n} \int_{\partial B_j} f(\zeta) d\zeta
    = \int_{\partial D} f(\zeta) d\zeta - \sum_{j=1}^{n} 2 \pi i \res(f, a_j)
  $$
\end{proof}

\begin{theorem}
  Przy powyższych założeniach, jeśli $ f(z) = \sum_{n=-\infty}{\infty} c_n (z-a)^n $ w $ V $, to $ \res(f, a) = c_{-1} $.
\end{theorem}

\begin{corollary}
  Niech funkcja $ f $ holomorficzna w otoczeniu pierścieniowym punktu $ a \in \mathbb{C} $. Wtedy jeśli $ a $ jest:
  \begin{enumerate}
    \item punktem pozornie osobliwym funkcji $f$, to $ \res(f, a) = 0 $ (na odwrót nie!);
    \item biegunem krotności $ 1 $ funkcji $ f $, to $ \res(f, a) = \lim_{z \to a} (z-a)f(z) $.
    W szczególności, jeśli $ f(z) = \frac{\phi(z)}{\psi(z)} $, $ \phi(a) \neq 0 $, $ \phi(a) = 0 $, $ \phi'(a) \neq 0 $, to $ \res(f, a) = \frac{\phi(z)}{\psi'(z)} $;
    \item biegunem krotności $k$ funkcji $f$, to $ \res(f, a) = \frac{1}{(k-1)!} \lim_{z \to a} \frac{\partial^{k-1}}{\partial z^{k-1}}\left( (z-a)^k f(z) \right)$.
  \end{enumerate}
  
\end{corollary}

\begin{definition}
  Jeśli funkcja $ f = \sum_{n=-\infty}^{\infty} c_n z^n $, $ \abs{z} > r $ ma w $ \infty $ izolowany punkt osobliwy, to
  $$
    \res(f, \infty) = \frac{1}{2 \pi i} \int_{\gamma_R^{-1}} f(z) dz,
    \qquad R > r
  $$
\end{definition}

\begin{remark}
   Ponieważ wyrazy z ujemnymi potęgami wchodzą w skład regularnej (a nie głównej) części rozwinięcia Laurenta funkcji w nieskończoności, to residuum w nieskończoności może nie być równe $0$ w punkcie regularnym.
\end{remark}

\begin{theorem}[O pełnej sumie residuów]
  Jeśli funkcja $f$ jest holomorficzna w $ \mathbb{C} $ za wyjątkiem skończonej liczby punktów osobliwych $ a_k $,
  to suma jej residuów w punktach $ a_k $ i w $ \infty $ jest równa zeru:
  $$
    \res(f, \infty) + \sum_k \res(f, a_k) = 0
  $$
\end{theorem}

\subsection{Dalsze własności funkcji holomorficznych i meromorficznych}

\begin{definition}[Pochodna logarytmiczna]
  Pochodną logarytmiczną funkcji $f$ nazywamy funkcję $ \frac{f'}{f} $.
\end{definition}

Jeśli $ f $ meromorficzna w pewnym obszarze $G$ i nie znika wszędzie w $G$,
to jej pochodna logarytmiczna jest meromorficzna i $ \res(\frac{f'}{f}, z_0) = \text{krotność } z_0 $,
gdzie $ z_0 $ jest zerem lub biegunem $f$.

% INFO: Ominięte dwa przykłady

\begin{theorem}[Zasada argumentu]
  Niech $\gamma$ kawałkami gładka krzywa Jordana (bez samoprzecięć).
  Niech $ D \subset \mathbb{C} $ obszar ograniczony przez $ \gamma $.
  Niech funkcja $f$ meromorficzna w $ G \supset \closure{D} $ ma $ Z(f) $ zer (z krotnościami) i $ B(f) $ biegunów (z krotnościami), i nie ma żadnych na $ \partial D $.
  Wtedy $ Z(F) - B(f) = \frac{1}{2 \pi} \Delta_{\gamma} arg f $.
\end{theorem}

\begin{proof}
  $ Z(f) - B(f) $ jest sumą residuów w zerach i biegunach w $ D $: % TODO: Dlaczego dokładnie?
  $$
    Z(f) - B(f)
    = \frac{1}{2 \pi} \int_{\gamma} \frac{f'(z)}{f(z)} dz
    = \frac{1}{2 \pi} \int_{0}^{2 \pi} \frac{f'(\gamma(t))}{f(\gamma(t))} dt
    = \frac{1}{2 \pi} \int_{0}^{2 \pi} \frac{(f \circ \gamma)'(t)}{(f \circ \gamma)(t)} dt
    = \frac{1}{2 \pi} \int_{f \circ \gamma} \frac{1}{z} dz
    = \frac{i \Delta_{\gamma} arg f}{2 \pi}
  $$
\end{proof}

\begin{theorem}[Rouchego]
  Niech $\gamma$ kawałkami gładka krzywa Jordana (bez samoprzecięć).
  Niech $ D \subset \mathbb{C} $ obszar ograniczony przez $ \gamma $.
  Niech $f$, $g$ holomorficzne w obszarze $ G \supset \closure{D} $ oraz $ \abs{f} > \abs{g} $ na $ \partial D $.
  Wtedy $f$ i $f+g$ mają tyle samo zer.
\end{theorem}

% TODO: Przydałby się obrazek z (1 + f/g)

\begin{proof}
  Możemy przedstawić $ f+g $ jako $ f(1+\frac{g}{f}) $.
  Część w nawiasie po drodze $ \partial D $ zawiera się w otwartym kole jednostkowym wokół $ (1, 0) $, więc nie okrąża zera i ma zerowy przyrost argumentu.
  Z holomorficzności $f+g$ nie ma biegunów.
  \begin{align*}
    Z(f+g) &
    = \frac{1}{2 \pi} \Delta_{\gamma} arg(f+g)
    = \frac{1}{2 \pi} \Delta_{\gamma} arg \left( f\left(1+\frac{g}{f}\right) \right) \\ &
    = \frac{1}{2 \pi} \left( \Delta_{\gamma} arg f + \Delta_{\gamma} arg\left(1+\frac{g}{f}\right) \right)
    = \frac{1}{2 \pi} \left( \Delta_{\gamma} arg f + 0) \right)
  \end{align*}
\end{proof}

% INFO: Ominięte dwa przykłady

\begin{theorem}[Hurwitza]
  Niech $f$ ma zero krotności $n$ w punkcie $ z_0 \in D $, $D$ obszar.
  Niech $ (f_n) $ ciąg funkcji holomorficznych zbieżny niemal jednostajnie do $f$.
  Wtedy istnieje $ \rho > 0 $ takie, że ddd. $ k $ $ f_k $ ma dokładnie $n$ zer w kole $ \abs{z - z_0} < \rho $ (licząc krotności).
\end{theorem}

\begin{proof}
  Z tw. Weierstrassa o szeregach funkcji holomorficznych wynika, że $f$ holomorficzna w $D$.
  Dobierzmy $ \rho > 0 $ tak, że $ K = K(z_0, \rho) \subset D $ i $ \forall z \in K\backslash\{ z_0 \}: f(z) \neq 0 $.
  Niech $ \delta = \min_{\partial K} \abs{f(z)}$.
  Skoro $ f_k $ zbiega do $ f $, to ddd. $k$ mamy $ \abs{f_k(z) - f(z)} < \delta $ na $ \partial K $.
  Z tw. Rouchego wynika, że $ f_k = (f_k - f) + f $ ma tyle samo zer na $K$ co $ f $.
\end{proof}

\begin{corollary}
  Jeśli ciąg $(f_k)$ funkcji holomorficznych i różnowartościowych w obszarze $D$ jest zbieżny niemal jednostajnie do funkcji $f$ różnej od stałej,
  to funkcja $f$ jest różnowartościowa w $D$.
\end{corollary}

\begin{proof}
  Załóżmy przeciwnie, że $f$ nie jest różnowartościowa w $D$ i przyjmuje tę samą wartość w $z_1$ i $z_2$.
  Rozpatrzmy ciąg $ g_k(z) = f_k(z) - f_k(z_2) $, zbieżny niemal jednostajnie do $ g(z) = f(z) - f(z_2) $.
  $ g(z) $ ma w $ z_1 $ zero pewnej krotności.
  Dobierzmy $ \rho $ tak, że $ z_2 \notin K(z_1, \rho) \subset D $.
  Z tw. Hurwitza ddd $k$ funkcje $g_k$ mają w tym kole zero, więc nie są różnowartościowe.
\end{proof}

\begin{definition}[Funkcja $p$-krotna]
  Niech $f$ holomorficzna w $z_0$ i $w_0=f(z_0)$.
  Funkcja $f$ jest $p$-krotna w otoczeniu $z_0$, jeśli
  $ \exists\enskip r > 0 \quad\forall\enskip 0 < \delta < r \quad\exists\enskip \eta > 0 \quad\forall\enskip w \in K(w_0, \eta)\backslash\{w_0\}: f(?) = w $
  w dokładnie $p$ różnych punktach $ K(z_0, \delta) $.
  %$$ \Forall{n \in \mathbb{N}_0} \Forall{a \in D} f^{(n)}(a) = \frac{n!}{2 \pi i} \int_{\partial D} \frac{f(\zeta)}{(\zeta - a)^{n+1}} d\zeta $$
\end{definition}

\begin{theorem}
  Jeśli $z_0$ jest $p$-krotnym zerem funkcji $f(z) - w_0$, to funkcja $f$ jest $p$-krotna w otoczeniu punktu $z_0$.
\end{theorem}

\begin{proof}
  Zera funkcji holomorficznych są izolowane, więc istnieje otoczenie $ U = \{z \in \mathbb{C}: 0 < \abs{z - z_0} < r \} $, w którym $ f(z) - w_0 $ oraz $ f'(z) $ nie mają zer.
  Weźmy $ 0 < \delta < r $ i niech $ \eta = \min_{z \in C(z_0, \delta)} \abs{f(z) - w_0} $.
  
  Niech $ w_1 \neq w_0 $ będzie dowolnym punktem koła $ K(w_0, \eta) $.
  Zdefiniujmy $ F(z) = f(z) - w_0 $ i $ G(z) = w_0 - w_1 $.
  Na okręgu $ C(z_0, \delta) $ zachodzi $ \abs{G(z)} < \eta \geq \abs{F(z)} $.
  Stąd z tw. Rouchego wynika, że $ F + G = f - w_1 $ ma w $ K(z_0, \delta) $ tyle samo zer co $ F $, czyli $p$.
  Wszystkie zera są jednokrotne, bo $ f'(z) \neq 0 $ w $ K(z_0, \delta)\backslash\{z_0\} $.
\end{proof}

\begin{corollary}[Tw. o zachowaniu obszaru]
  Funkcja holomorficzna różna od stałej przekształca obszar na obszar.
\end{corollary}

\begin{proof}
  Niech $ w_0 = f(z_0) $, wtedy $ z_0 $ jest zerem pewnej krotności funkcji $ f(z) - w_0 $.
  Stąd funkcja $f$ jest p-krotna w otoczeniu $z_0$.
  Stąd dla każdego dysku $ D(z_0, r) $ istnieje $ D(w_0, \epsilon) $ taki, że każdy punkt $ w_1 \in D(w_0, \epsilon) $ ma w przeciwobrazie $p$ różnych punktów w $ D(z_0, r) $.
\end{proof}

\begin{corollary}[Tw. o funkcji odwrotnej]
  Jeśli $ f'(z_0) \neq 0 $ lub $z_0$ jest jednokrotnym biegunem $ f $, to w pewnym otoczeniu punktu $z_0$ funkcja $f$ ma funkcję odwrotną.
\end{corollary}

\begin{proof}
  Jeśli $ f'(z_0) \neq 0 $ to $ z_0 $ jest jednokrotnym zerem funkcji $ f - f(z_0) $, więc $f$ jest jednokrotna w pewnym otoczeniu $U$ punktu $z_0$, więc ma f. odwrotną.
  Niech $ g = f^{-1} $.
  Pokazujemy, że $g$ ma pochodną w każdym punkcie $ w_1 \in f(U) $.
  Oznaczmy $ z_1 = g(w_1) $.
  Zauważmy, że $g$ jest ciągła, w p.p. ciąg $ g(w_n) $ miałby punkt skupienia różny od $ z_1 $ a wtedy $ f $ nie byłaby jednokrotna (przyjmowała by wartość $ z_1 $ w dwóch punktach).
  Iloraz $ \frac{g(w) - g(w_1)}{w - w_1} = \frac{z - z_1}{f(z) - f(z_1)} \to \frac{1}{f'(z_1)} $ po $ w \to w_1 $.
  Jeśli $ z_0 $ jest jednokrotnym biegunem funkcji $f$, to $ h = 1/f $ ma zero krotności $1$, więc $h^{-1}$ istnieje, zatem i $f^{-1}$. 
\end{proof}

\section{Dalsze własności odzworowań konforemnych}

\begin{definition}
  Obszary $ \extcomplex $, $ \mathbb{C}$, $ U = K(0, 1) $ nazywamy obszarami podstawowymi.
\end{definition}

\begin{theorem}
  Wszystkie automorfizmy obszarów podstawowych są homografiami.
\end{theorem}

\begin{proof}
  (1A) Niech $ \phi : \mathbb{C} \to \mathbb{C} $, $ z_0 \neq \infty $, $ \phi(z_0) = \infty $.
  Wtedy $ \phi $ jest holomorficzna poza $ z_0 $, a w nim ma biegun.
  Pokażemy, że krotność tego bieguna to $1$.

  Weźmy $ \psi(z) = \frac{1}{\phi(z)} $.
  $ \psi $ musi mieć zero krotności $1$ w punkcie $z_0$, w p.p. $ \phi $ nie byłaby jednokrotna w pierścieniowym otoczeniu $ z_0 $.
  Stąd $ \phi = \frac{a}{z-z_0} + g(z) $, gdzie $g$ jest holomorficzna na $ \mathbb{C} $.
  Mamy $ \lim_{z \to \infty} g(z) = \lim_{z \to \infty} \phi(z) = \phi(\infty) \in \mathbb{C} $,
  więc z tw. Liouville'a $ g(z) = c $, a stąd $ \phi $ jest homografią.

  (1B) Niech teraz $ z_0 = \infty $.
  Wtedy $ g(z) = \frac{1}{\phi(1/z)} $ ma jednokrotne zero w $ 0 $.
  Stąd $ \phi(1/z) $ ma biegun krotności $1$ w $0$, więc $ \phi(1/z) = \frac{a}{z} + g(z) $.
  Podobnie jak poprzednio, przy $ z \to \infty $ mamy $ g(z) \to c \in \mathbb{C} $.
  Stąd $ g(z) = c $, więc $ \phi = az + c $.

  (2) Pokażemy, że $ \phi : \mathbb{C} \to \mathbb{C} $ można rozszerzyć do $ \extcomplex $ kładąc $ \phi(\infty) = \infty $.
  Niech $ z_n \to \infty $.
  Jeżeli $ z_n $ nie zbiega do $ \infty $, to istniałby w nim podciąg zbieżny do skończonej liczby, a wtedy przekształcenie odwrotne byłoby nieciągłe.
  Skoro jest zbieżność do $ \infty $, to $ \frac{1}{\phi} $ jest holomorficzny i ma w $\infty$ osobliwość pozorną.
  Można ją uzupełnić, więc $ \phi $ też można uzupełnić.
  Więc $ \phi $ jest homografią z punktem stałym $ \phi(\infty) = \infty $, skąd $ \phi(z) = az + c $.

  (3) Niech $ \phi : U \to U $.
  Oznaczmy $ w_0 = \phi(0) $.
  Patrzymy na $ h:U \to U $, $ h(w) = \frac{w-w_0}{1 - \closure{w_0}w} $, która przeprowadza $w_0$ na $0$.
  Wtedy $ f = h \circ \phi $ jest automorfizmem $U$ spełniającym $ f(0) = 0 $.
  Możemy skorzystać z Lematu Schwarza, więc $ \abs{f(z)} \leq \abs{z} $.
  Podobnie dla $ f^{-1} $ jest $\abs{f^{-1}(z)} \leq \abs{z} $.
  Musi więc być $ \abs{f(z)} = \abs{z} $.
  Znowu z lematu Schwarza $ f(z) = e^{it}$, więc $f$ jest homografią, więc także i $\phi$.
\end{proof}

\begin{corollary}
  Aut $ \extcomplex = \{ f(z) = \frac{az+b}{cz+d}: a,b,c,d \in \mathbb{C}, ad-bc \neq 0 \}$, \\
  Aut $ \mathbb{C} = \{ f(z) = az+b: a,b \in \mathbb{C}, a \neq 0 \}$, \\
  Aut $ U = \{ f(z) = e^{it}frac{z-a}{1 - \closure{a}{z}}: a \in \mathbb{C}, t \in \mathbb{R}, \abs{a} < 1 \} $.
\end{corollary}

\begin{theorem}
  Obszary $ \extcomplex $, $ \mathbb{C}$, $ U = K(0, 1) $ są parami niehomeomorficzne.
\end{theorem}

\begin{proof}
  $ \extcomplex $ jest zwarty, a reszta nie.
  Gdyby było $ \alpha: \mathbb{C} \to U $, to z tw. Liouville'a $ \alpha(z) = c $, sprzeczność.
\end{proof}

% TODO: Tw. Riemanna o odwzorowaniach

\end{document}
