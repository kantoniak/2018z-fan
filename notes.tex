% Document setup
\documentclass[11pt]{article}
\usepackage[a4paper,margin=1.75cm]{geometry}

% Language and encoding
\usepackage[T1]{fontenc}
\usepackage[utf8]{inputenc}
\usepackage{polski}

% Imports
\usepackage{amsmath}
\usepackage{amssymb}
\usepackage{amsthm}

% Styling
\usepackage{lmodern}
\setlength{\parindent}{0cm} % No paragraph indents for the first line
\usepackage{secdot} % Add dot after section number
\newcommand{\extcomplex}{\overline{\mathbb{C}}} % Extended complex plane
\newcommand{\abs}[1]{|#1|} % Absolute value
\newcommand{\norm}[1]{\left\lVert#1\right\rVert} % Norm

% Theorems and definitions
\theoremstyle{definition}
\newtheorem*{definition}{Definicja}

% Document meta
\title{Funkcje analityczne}
\makeatletter

\begin{document}

% Title
{\huge\bfseries\@title\par}
\vspace{0.1cm}
{\Large Notatki z semestru zimowego 2018/2019\par}

\section{Szeregi o wyrazach zespolonych}

\begin{definition}
  Szereg zespolony $\sum_{n=1}^{\infty} a_{n}$ jest zbieżny do sumy $ s \in \mathbb{C} $ wtedy i tylko wtedy, gdy $ \sum_{n=1}^{k} a_{n} $ ma granicę $ s $ przy $ k\to\infty $.
\end{definition}

\begin{definition}[Granica szeregu funkcyjnego]
  Niech $ K \subset \extcomplex $ będzie dowolnym zbiorem. Niech $ f_{n} \colon K \to \mathbb{C} $, $ \norm{f}_{K} = \sup_{K}\abs{f(z)} $. Wówczas $ (f_{n}) $ jest zbieżny jednostajnie do $ f $ wtedy i tylko wtedy, gdy
  $$ \lim_{k\to\infty} \norm{f - \sum_{n=1}^{k} f_{n}}_{K} = 0 $$.
\end{definition}

% TODO: Własność: całkowanie szeregu jednostajnie zbieżnego
% TODO: Właśność: Kryterium Weierstrassa jednostajnej zbieżności
% TODO: Twierdzenie o rozwijaniu funkcji holomorficznej w szereg potęgowy
 
\end{document}
